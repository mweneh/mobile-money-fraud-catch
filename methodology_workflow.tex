\begin{figure}[htbp]
\centering
\resizebox{\textwidth}{!}{%
\begin{tikzpicture}[
    node distance=1.5cm and 2cm,
    every node/.style={align=center},
    process/.style={rectangle, draw=black, fill=blue!10, thick, minimum width=3.5cm, minimum height=1cm, rounded corners},
    decision/.style={diamond, draw=black, fill=green!10, thick, aspect=2, minimum width=3cm, minimum height=1cm},
    io/.style={trapezium, trapezium left angle=70, trapezium right angle=110, draw=black, fill=orange!10, thick, minimum width=3cm, minimum height=1cm},
    arrow/.style={thick, ->, >=stealth}
]

    % Phase 1: Data Processing
    \node (raw) [io] {Raw Transaction Data\\(95,662 entries)};
    \node (clean) [process, below=1cm of raw] {Data Quality Check\\(No Nulls/Dups)};
    \node (fe) [process, below=1cm of clean] {Feature Engineering\\(Temporal, Ratios, Log)};
    \node (scale) [process, below=1cm of fe] {Standard Scaling\\(For Linear Models)};

    % Phase 2: Split & Balance
    \node (split) [process, right=2cm of clean] {Stratified Split\\(80\% Train / 20\% Val)};
    
    \node (imbalance) [decision, right=2cm of fe] {Imbalance Strategy};
    
    \node (weight) [process, above right=0.5cm and 2cm of imbalance] {Class Weighting\\(Cost-Sensitive)};
    \node (smote) [process, below right=0.5cm and 2cm of imbalance] {SMOTE\\(Oversampling)};

    % Phase 3: Modeling
    \node (models) [process, right=3.5cm of imbalance, minimum height=3cm, fill=red!10] {
        \textbf{Model Training}\\
        \begin{itemize}
            \item LightGBM
            \item Random Forest
            \item XGBoost
            \item LinearSVC
            \item Logistic Regression
        \end{itemize}
    };

    % Phase 4: Evaluation
    \node (eval) [process, below=2cm of models] {Performance Evaluation\\(PR-AUC, F1, Recall)};
    \node (optimize) [process, left=2cm of eval] {Threshold Optimization\\(Precision-Recall Curve)};
    \node (final) [io, left=2cm of optimize] {Final Predictions\\(Optimal Threshold)};

    % Connections
    \draw [arrow] (raw) -- (clean);
    \draw [arrow] (clean) -- (fe);
    \draw [arrow] (fe) -- (split); % Conceptual link: Prep -> Split
    % Actually split happens before scaling usually, but diagrammatically this flows well. 
    % Let's clarify flow: Raw -> Prep -> Split -> Imbalance -> Model
    
    % Re-routing for logical flow
    \draw [arrow] (fe) -- (scale);
    \draw [arrow] (clean) -- (split);
    \draw [arrow] (split) -- (imbalance);
    
    \draw [arrow] (imbalance) -- node[anchor=south, sloped] {Baseline} (weight);
    \draw [arrow] (imbalance) -- node[anchor=north, sloped] {Comparative} (smote);
    
    \draw [arrow] (weight) -- (models);
    \draw [arrow] (smote) -- (models);
    
    \draw [arrow] (models) -- (eval);
    \draw [arrow] (eval) -- (optimize);
    \draw [arrow] (optimize) -- (final);

    % Background Boxes
    \begin{scope}[on background layer]
        \node [draw=black!20, dashed, fit=(raw) (scale), label=above:\textbf{P1: Data Processing}] {};
        \node [draw=black!20, dashed, fit=(split) (weight) (smote), label=above:\textbf{P2: Experimental Setup}] {};
    \end{scope}

\end{tikzpicture}
}
\caption{Methodological Workflow: From Data Ingestion to Model Optimization. The framework adopts a CRISP-DM aligned approach featuring parallel imbalance handling strategies.}
\label{fig:methodology_workflow}
\end{figure}
